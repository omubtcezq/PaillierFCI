%%%%%%%%%%%%%%%%%%%%%%%%%%%%%%%%%%%%%%%%%%%%%%%%%%%%%%%%%%%%%%%%%%%%%%%%%%%%%%%%
%2345678901234567890123456789012345678901234567890123456789012345678901234567890
%        1         2         3         4         5         6         7         8

% Comment this line out if you need a4paper
\documentclass[letterpaper, 10 pt, conference]{ieeeconf}  

% Use this line for a4 paper
%\documentclass[a4paper, 10pt, conference]{ieeeconf}

% This command is only needed if you want to use the \thanks command
\IEEEoverridecommandlockouts

% Needed to meet printer requirements.
\overrideIEEEmargins

%In case you encounter the following error:
%Error 1010 The PDF file may be corrupt (unable to open PDF file) OR
%Error 1000 An error occurred while parsing a contents stream. Unable to analyze the PDF file.
%This is a known problem with pdfLaTeX conversion filter. The file cannot be opened with acrobat reader
%Please use one of the alternatives below to circumvent this error by uncommenting one or the other
%\pdfobjcompresslevel=0
%\pdfminorversion=4

% See the \addtolength command later in the file to balance the column lengths
% on the last page of the document

% Imports
\usepackage{amsmath}
\usepackage{amsfonts}
%\usepackage{amsthm} % Errors from \proof when used with this ieeeconf
% Local
\usepackage{local_macros/isasmathmacros}

% Environments
%\theoremstyle{definition} % Doesn't exist when used with this ieeeconf
\newtheorem{definition}{Definition}[section]
\newtheorem{theorem}{Theorem}[section]

%\theoremstyle{remark} % Doesn't exist when used with this ieeeconf
%\newtheorem*{remark}{Remark} % Doesn't exist when used with this ieeeconf
\newtheorem{remark}{Remark}

\title{\LARGE \bf
Encrypted Fast Covariance Intersection on the Cloud Without Leaking Weights
}

\author{Marko Ristic$^{1}$ and Benjamin Noack$^{1}$% <-this % stops a space
\thanks{$^{1}$Marko Ristic and Benjamin Noack are with the Autonomous Multisensor Systems Group (AMS), Institute for Intelligent Cooperating Systems (ICS), Otto von Guericke University (OVGU), Magdeburg, Germany {\tt\small \{marko.ristic, benjamin.noack\}@ovgu.de}}%
}

\begin{document}

\maketitle
\thispagestyle{empty}
\pagestyle{empty}

%%%%%%%%%%%%%%%%%%%%%%%%%%%%%%%%%%%%%%%%%%%%%%%%%%%%%%%%%%%%%%%%%%%%%%%%%%%%%%%%
% 
%        d8888 888888b.    .d8888b.  
%       d88888 888  "88b  d88P  Y88b 
%      d88P888 888  .88P  Y88b.      
%     d88P 888 8888888K.   "Y888b.   
%    d88P  888 888  "Y88b     "Y88b. 
%   d88P   888 888    888       "888 
%  d8888888888 888   d88P Y88b  d88P 
% d88P     888 8888888P"   "Y8888P"  
%                                    
%                                    
%                                    
% 
\begin{abstract}
    State estimate fusion is a common requirement in distributed sensor networks and can be complicated by untrusted participants or network eavesdroppers. We present a method for computing the common Fast Covariance Intersection fusion algorithm on an untrusted cloud without disclosing individual estimates or the fused result. In an existing solution to this problem, fusion weights corresponding to the sensor estimate errors are leaked to the cloud in order to perform the fusion. In this work, we present a method that guarantees no leakage at the cloud by requiring an additional computation step by the party querying the cloud for the fused result. The Paillier encryption scheme is used to homomorphically compute seperate parts of the computation that can be combined after decryption. This encrypted Fast Covariance Intersection algorithm can be used in scenarios where the fusing cloud is not trusted and relative sensor performances must remain confidential.
\end{abstract}

%%%%%%%%%%%%%%%%%%%%%%%%%%%%%%%%%%%%%%%%%%%%%%%%%%%%%%%%%%%%%%%%%%%%%%%%%%%%%%%%

% 
% 8888888 888b    888 88888888888 8888888b.   .d88888b.  
%   888   8888b   888     888     888   Y88b d88P" "Y88b 
%   888   88888b  888     888     888    888 888     888 
%   888   888Y88b 888     888     888   d88P 888     888 
%   888   888 Y88b888     888     8888888P"  888     888 
%   888   888  Y88888     888     888 T88b   888     888 
%   888   888   Y8888     888     888  T88b  Y88b. .d88P 
% 8888888 888    Y888     888     888   T88b  "Y88888P"  
%                                                        
%                                                        
%                                                        
% 
\section{Introduction}\label{sec:introduction}
Data fusion and distributed state estimation have long been active fields of research and continue to find many applications in modern systems today \cite{andersonOptimalFiltering1979,simonOptimalStateEstimation2006}. Methods relying on the Kalman filter and derivatives \cite{haugBayesianEstimationTracking2012} have become particularly prevalent in this area due to their recursive structure and suitability to modelling estimate cross-correlations typically required for data fusion \cite{mutambaraDecentralizedEstimationControl1998,ligginsDistributedDataFusion2012}. The handling of these cross-correlations is especially important when consistent or optimal fusion is desired \cite{bar-shalomTracktotrackCorrelationProblem1981,sunMultisensorOptimalInformation2004} and presents a key challenge in data fusion when they are not known in advance. To overcome this, some methods propagate cross-correlations through time at the cost of repeated reconstruction \cite{steinbringOptimalSamplebasedFusion2016} and typically add local computational complexity. Alternative methods approximate error cross-correlations with conservative suboptimal estimates to provide consistent fusion instead \cite{julierNondivergentEstimationAlgorithm1997,noackDecentralizedDataFusion2017,niehsenInformationFusionBased2002}. One such popular method is Covariance Intersection \cite{julierNondivergentEstimationAlgorithm1997} and it computationally inexpensive approximation, Fast Covariance Intersection \cite{niehsenInformationFusionBased2002}. These methods minimise fusion estimate error given possible conservative estimates and are popular due to their compatibility with the information form of the Kalman filter \cite{mutambaraDecentralizedEstimationControl1998,pfaffInformationFormDistributed2017}.

While effective and often efficient, these data fusion solutions have traditionally been performed on closed networks with trusted participants and inherently imply a trust between sensors and estimators. In recent years, the ubiquity of distributed public networks has seen the additional challenges of security and privacy become increasingly relevant in distributed sensing environments and has led to an active field of reserach in providing security during data processing and fusion tasks \cite{renSecurityChallengesPublic2012,brennerSecretProgramExecution2011}. While ensuring transimitted information is kept secret from eavesdroppers on untrusted networks is achievable with common private and public key encryption schemes \cite{katzIntroductionModernCryptography2008}, tasks involving untrusted participants during computations present a greater challenge. Here, partial computations on encrypted data or the leakage of intermediate results are often required for arriving at the final result \cite{risticSecureFastCovariance2021,shiPrivacyPreservingAggregationTimeSeries2011}. One common tool for achieving these requirements is homomorphic encryption \cite{gentryFullyHomomorphicEncryption2009,paillierPublicKeyCryptosystemsBased1999} which allows operations to be computed on encrypted numbers without decryption. The Pailler encryption scheme \cite{paillierPublicKeyCryptosystemsBased1999} allows homomorphic addition and is a frequent choice in fusion application due to its applicability, simplicity and provable security. In \cite{alanwarPrOLocResilientLocalization2017}, the Paillier scheme is used to perform non-Bayesian measurement fusion with range-sensors without disclosing individual sensor locations or measurements, while in \cite{aristovEncryptedMultisensorInformation2018}, it is used in to achieve similar security with a Kalman filter when measurements are linear and sensors form a heirarchical network. When paired with additional schemes that introduce some leakage, tasks in a wider variety of scenarios can often be solved as well. \cite{alexandruEncryptedCooperativeControl2019} uses private weighted sum aggregation together with the Paillier scheme to compute decentralised local control inputs while leaking only the sum of weighted neightbouring states, while in \cite{risticSecureFastCovariance2021}, an untrusted cloud can use order-revealing encryptions to compute and leak relative sensor accuracies allowing the homomorphic computation of the Covariance Intersection fusion algorithm.

In this work, we consider stochastic state estimate fusion on an untrusted cloud similar to \cite{risticSecureFastCovariance2021}, but aim to fuse estimates with no leakage at the cloud. Our contribution is a novel method for computing the Fast Covariance Intersection algorithm using only the Paillier encryption scheme and leaking no intermediate information to the untrsuted cloud. While in some use-cases, leaked relative sensor accuracies can be useful for prioritising sensor communications, this leakage inherently informs the cloud about sensor performances over time, disclosing the times when sensors are out of range or non-functioning and implicitly leaking information about the fused estimate. Therefore, the presented method finds application when all sensor information is considered sensitive and when the fusing cloud is not trusted.

Section~\ref{sec:problem} introduces the formal data fusion and security goals and section~\ref{sec:prelims} gives relevant preliminaries. Section~\ref{sec:encrypted_fci} introduces our method, before sections~\ref{sec:security} and \ref{sec:simulation} analyse its security and simulation results, respectively. Concluding remarks and future work are discussed in section~\ref{sec:conclusion}.

\subsection{Notation}
% TODO fill


% 
% 8888888b.  8888888b.   .d88888b.  888888b.   
% 888   Y88b 888   Y88b d88P" "Y88b 888  "88b  
% 888    888 888    888 888     888 888  .88P  
% 888   d88P 888   d88P 888     888 8888888K.  
% 8888888P"  8888888P"  888     888 888  "Y88b 
% 888        888 T88b   888     888 888    888 
% 888        888  T88b  Y88b. .d88P 888   d88P 
% 888        888   T88b  "Y88888P"  8888888P"  
%                                              
%                                              
%                                              
% 
\section{Problem Statement}\label{sec:problem}


% 
% 8888888b.  8888888b.  8888888888 888      8888888 888b     d888 
% 888   Y88b 888   Y88b 888        888        888   8888b   d8888 
% 888    888 888    888 888        888        888   88888b.d88888 
% 888   d88P 888   d88P 8888888    888        888   888Y88888P888 
% 8888888P"  8888888P"  888        888        888   888 Y888P 888 
% 888        888 T88b   888        888        888   888  Y8P  888 
% 888        888  T88b  888        888        888   888   "   888 
% 888        888   T88b 8888888888 88888888 8888888 888       888 
%                                                                 
%                                                                 
%                                                                 
% 
\section{Preliminaries}\label{sec:prelims}

\subsection{Fast Covariance Intersection}\label{subsec:fci}

\subsection{Paillier Encryption Scheme}\label{subsec:paillier}

\subsection{Integer Encoding for Homomorphic Encryption}\label{subsec:encoding}


% 
% 888b     d888 8888888888 88888888888 888    888  .d88888b.  8888888b.  
% 8888b   d8888 888            888     888    888 d88P" "Y88b 888  "Y88b 
% 88888b.d88888 888            888     888    888 888     888 888    888 
% 888Y88888P888 8888888        888     8888888888 888     888 888    888 
% 888 Y888P 888 888            888     888    888 888     888 888    888 
% 888  Y8P  888 888            888     888    888 888     888 888    888 
% 888   "   888 888            888     888    888 Y88b. .d88P 888  .d88P 
% 888       888 8888888888     888     888    888  "Y88888P"  8888888P"  
%                                                                        
%                                                                        
%                                                                        
% 
\section{Encrypted Fast Covariance Intersection}\label{sec:encrypted_fci}


% 
%  .d8888b.  8888888888 .d8888b.  
% d88P  Y88b 888       d88P  Y88b 
% Y88b.      888       888    888 
%  "Y888b.   8888888   888        
%     "Y88b. 888       888        
%       "888 888       888    888 
% Y88b  d88P 888       Y88b  d88P 
%  "Y8888P"  8888888888 "Y8888P"  
%                                 
%                                 
%                                 
% 
\section{Security Analysis}\label{sec:security}


% 
%  .d8888b. 8888888 888b     d888 
% d88P  Y88b  888   8888b   d8888 
% Y88b.       888   88888b.d88888 
%  "Y888b.    888   888Y88888P888 
%     "Y88b.  888   888 Y888P 888 
%       "888  888   888  Y8P  888 
% Y88b  d88P  888   888   "   888 
%  "Y8888P" 8888888 888       888 
%                                 
%                                 
%                                 
% 
\section{Simulation and Results}\label{sec:simulation}


% 
%  .d8888b.   .d88888b.  888b    888  .d8888b.  
% d88P  Y88b d88P" "Y88b 8888b   888 d88P  Y88b 
% 888    888 888     888 88888b  888 888    888 
% 888        888     888 888Y88b 888 888        
% 888        888     888 888 Y88b888 888        
% 888    888 888     888 888  Y88888 888    888 
% Y88b  d88P Y88b. .d88P 888   Y8888 Y88b  d88P 
%  "Y8888P"   "Y88888P"  888    Y888  "Y8888P"  
%                                               
%                                               
%                                               
% 
\section{Conclusion}\label{sec:conclusion}


%%%%%%%%%%%%%%%%%%%%%%%%%%%%%%%%%%%%%%%%%%%%%%%%%%%%%%%%%%%%%%%%%%%%%%%%%%%%%%%%

% This command serves to balance the column lengths
% on the last page of the document manually. It shortens
% the textheight of the last page by a suitable amount.
% This command does not take effect until the next page
% so it should come on the page before the last. Make
% sure that you do not shorten the textheight too much.
%\addtolength{\textheight}{-11.05cm}

%%%%%%%%%%%%%%%%%%%%%%%%%%%%%%%%%%%%%%%%%%%%%%%%%%%%%%%%%%%%%%%%%%%%%%%%%%%%%%%%

% 
% 8888888b.  8888888888 8888888888 .d8888b.  
% 888   Y88b 888        888       d88P  Y88b 
% 888    888 888        888       Y88b.      
% 888   d88P 8888888    8888888    "Y888b.   
% 8888888P"  888        888           "Y88b. 
% 888 T88b   888        888             "888 
% 888  T88b  888        888       Y88b  d88P 
% 888   T88b 8888888888 888        "Y8888P"  
%                                            
%                                            
%                                            
% 
\bibliographystyle{IEEEtran}
\bibliography{bibliography/PaillierFCI}

\end{document}
